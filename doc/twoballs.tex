\documentclass[letterpaper,final,12pt,reqno]{amsart}

\usepackage[total={6.3in,9.2in},top=1.1in,left=1.1in]{geometry}

\usepackage{times,bm,bbm,empheq,verbatim,fancyvrb,graphicx}
\usepackage[dvipsnames]{xcolor}

\usepackage[kw]{pseudo}

\pseudoset{left-margin=15mm,topsep=5mm,idfont=\texttt}

\usepackage{tikz}
\usetikzlibrary{decorations.pathreplacing}

% hyperref should be the last package we load
\usepackage[pdftex,
colorlinks=true,
plainpages=false, % only if colorlinks=true
linkcolor=blue,   % ...
citecolor=Red,    % ...
urlcolor=black    % ...
]{hyperref}

\DefineVerbatimEnvironment{cline}{Verbatim}{fontsize=\small,xleftmargin=5mm}

\renewcommand{\baselinestretch}{1.05}

\newtheorem{lemma}{Lemma}

\newcommand{\Matlab}{\textsc{Matlab}\xspace}
\newcommand{\eps}{\epsilon}
\newcommand{\RR}{\mathbb{R}}

\newcommand{\grad}{\nabla}
\newcommand{\Div}{\nabla\cdot}
\newcommand{\trace}{\operatorname{tr}}

\newcommand{\hbn}{\hat{\mathbf{n}}}

\newcommand{\bb}{\mathbf{b}}
\newcommand{\be}{\mathbf{e}}
\newcommand{\bbf}{\mathbf{f}}
\newcommand{\bg}{\mathbf{g}}
\newcommand{\bn}{\mathbf{n}}
\newcommand{\bq}{\mathbf{q}}
\newcommand{\br}{\mathbf{r}}
\newcommand{\bu}{\mathbf{u}}
\newcommand{\bv}{\mathbf{v}}
\newcommand{\bw}{\mathbf{w}}
\newcommand{\bx}{\mathbf{x}}

\newcommand{\bV}{\mathbf{V}}
\newcommand{\bX}{\mathbf{X}}

\newcommand{\bxi}{\bm{\xi}}

\newcommand{\blambda}{\bm{\lambda}}
\newcommand{\bzero}{\bm{0}}

\newcommand{\rhoi}{\rho_{\text{i}}}
\newcommand{\ip}[2]{\left<#1,#2\right>}

\newcommand{\Rpr}{R_{\text{pr}}}
\newcommand{\Rin}{R_{\text{in}}}
\newcommand{\Rfw}{R_{\text{fw}}}


\begin{document}
\title[Two balls connected by a rod]{Two balls connected by a rod: \\ an index 3 differential-algebraic equations case study.}

\author{Ed Bueler}

\begin{abstract}
The problem of two equal-mass balls, rigidly connected by a massless rod, is described by an index-3 differential-algebraic equations (DAE) system.  Numerical solutions based on an index-2 stabilized form and implicit PETSc TS solvers are evaluated.
\end{abstract}

\maketitle

%\tableofcontents

\thispagestyle{empty}
\bigskip

\section{Equations of motion}

Consider the problem of two equal masses $m$, labeled ``$a$'' and ``$b$'', moving in the $(x,y)$ plane, with $x$ horizontal and $y$ vertical.  Their cartesian coordinates form a configuration column vector
    $$\bq(t) = \begin{bmatrix} q_1(t) \\ q_2(t) \\ q_3(t) \\ q_4(t) \end{bmatrix} = \begin{bmatrix} x_a(t) \\ y_a(t) \\ x_b(t) \\ y_b(t) \end{bmatrix}.$$
Denote their velocities by $\bv$, thus $v_1 = \dot x_a$, $v_2=\dot y_a$, $v_3 = \dot x_b$, and $v_4=\dot y_b$:
    $$\bv(t) = \dot\bq(t).$$

Now suppose the two masses move according to two kinds of forces.  First we have constant gravity acting vertically downwards.  Second, suppose the masses are connected by a rigid, massless rod with length $\ell$.  The locations of the two masses are now constrained to satisfy
    $$(x_a - x_b)^2 + (y_a - y_b)^2 = \ell^2$$
at any time.  Equivalently there is a scalar function of the coordinates which is identically zero:
\begin{equation}
g(\bq) = \frac{1}{2} \Big((q_1 - q_3)^2 + (q_2 - q_4)^2 - \ell^2\Big) = 0. \label{constraint}
\end{equation}
(The overall constant $\frac{1}{2}$ is chosen for convenience.)  Physically speaking, the rod exerts some tension or expansion force between the masses, which varies during the motion.  Denoting it by a scalar $\lambda$, of either sign, but positive when the rod is pulling the two masses together, what is $\lambda$?  Furthermore, from the resulting total forces, what are the equations of motion?

Newton's original laws are poorly-suited to describing the forces in such a constrained situation.  However, Hamilton's principle of least action \cite[equation (52.1)]{Lanczos1970} applies.  First, for unconstrained motion described by the configuration variables $\bq(t) \in \RR^n$ and velocities $\bv(t) = \dot\bq$, one defines the \emph{Lagrangian} $\mathcal{L}(\bq,\bv) = T(\bq,\bv) - U(\bq)$, the difference of kinetic and potential energy.  The principle then says that the motion $\bq(t)$ solves the Euler-Lagrange differential equation system ($i=1,\dots,n$):
\begin{equation}
\frac{d}{dt} \frac{\partial \mathcal{L}}{\partial v_i} = \frac{\partial \mathcal{L}}{\partial q_i}. \label{eulerlagrange}
\end{equation}

For constrained motion we may also use Hamilton's principle, but only after modifying the Lagrangian.  In general cases one assumes $\bg(\bq) \in \RR^k$ is a column vector of the constraints, i.e.~$\bg(\bq)=\bzero$, and one supposes a (column) vector of Lagrange multipliers $\blambda(t) \in \RR^k$.  This defines a new potential energy \cite[equation (58.2)]{Lanczos1970},
\begin{equation}
\tilde U(\bq,\blambda) = U(\bq) + \blambda^\top \bg(\bq), \label{extendedpotential}
\end{equation}
and a modified Lagrangian $\mathcal{L}(\bq,\bv,\blambda) = T(\bq,\bv) - \tilde U(\bq,\blambda)$.  Then the motion $\bq(t),\bv(t),\blambda(t)$ satisfies the system \eqref{eulerlagrange} along with the constraints.

Observe that, because the modified Lagrangian does not depend on $\dot\blambda$, one may recover the constraints by the notional equation $\bzero = d/dt(\partial \mathcal{L}/\partial \dot\blambda) = \partial \mathcal{L}/\partial\blambda = \bg(\bq)$.  In this sense the multipliers are treated as coordinates and the Euler-Lagrange equations now include the constraints.  Physically, however, $\blambda$ is the (vector) force which enforces the constraints.

In our two-masses problem with $n=4$ there is $k=1$ (scalar) constraint $g(\bq)$ and thus one Lagrange multiplier $\lambda$.  It is the rod tension which we seek!  Based on the usual formula for kinetic energy, independent of the cartesian coordinate $\bq$, and on gravitational potential energy, we have
\begin{equation}
T(\bv) = \frac{m}{2} \left(v_1^2+v_2^2+v_3^2+v_4^2\right), \qquad U(\bq) = m g_r \left(q_2+q_4\right), \label{energies}
\end{equation}
where $g_r>0$ is the acceleration of gravity.  Thus we define the modified Lagrangian
\begin{align}
\mathcal{L}(\bq,\bv,\lambda) &= T(\bq,\bv) - V(\bq) - \lambda g(\bq) \label{lagrangian} \\
  &= \frac{m}{2} \left(v_1^2+v_2^2+v_3^2+v_4^2\right) - m g_r \left(q_2+q_4\right) \notag \\
  &\qquad - \frac{\lambda}{2} \Big((q_1 - q_3)^2 + (q_2 - q_4)^2 - \ell^2\Big). \notag
\end{align}
The Euler-Lagrange system \eqref{eulerlagrange} is now a set of 4 equations:
\begin{align*}
m \dot v_1 &= - \lambda (q_1 - q_3) \\
m \dot v_2 &= - m g_r - \lambda (q_2 - q_4) \\
m \dot v_3 &= \lambda (q_1 - q_3) \\
m \dot v_4 &= - m g_r + \lambda (q_2 - q_4)
\end{align*}

In total, however, our constrained two-masses problem is a system of 9 first-order differential equations.  It includes the definition of the velocities and the constraint \eqref{constraint} itself:
\begin{align*}
  \dot q_1 &= v_1 \\
  \dot q_2 &= v_2 \\
  \dot q_3 &= v_3 \\
  \dot q_4 &= v_4 \\
m \dot v_1 &= - \lambda (q_1 - q_3) \\
m \dot v_2 &= - m g_r - \lambda (q_2 - q_4) \\
m \dot v_3 &= \lambda (q_1 - q_3) \\
m \dot v_4 &= - m g_r + \lambda (q_2 - q_4) \\
         0 &= \frac{1}{2} \Big((q_1 - q_3)^2 + (q_2 - q_4)^2 - \ell^2\Big)
\end{align*}

In anticipation of the theory in reference \cite{AscherPetzold1998}, wherein our type of system appears as equation (9.30), we prefer to write the system using vectors and matrices:
\begin{align}
\dot \bq &= \bv \notag \\
m \dot \bv &= \bbf - \lambda\, G(\bq)^\top \label{system} \\
0 &= g(\bq) \notag
\end{align}
Here the column vector $\bbf = [0,-mg_r,0,-mg_r]^\top$ is the constant force of gravity and $G$ is a $k\times n = 1\times n$ Jacobian matrix:
\begin{equation}
G(\bq) = \begin{bmatrix} \frac{\partial g}{\partial q_i} \end{bmatrix} = \begin{bmatrix} q_1-q_3, & q_2-q_4, & -(q_1-q_3), & -(q_2-q_4) \end{bmatrix}. \label{constraintjacobian}
\end{equation}
(Reference \cite{AscherPetzold1998} allows a mass matrix $M(\bq)$, but in our case $M(\bq) = mI$.)

System \eqref{system} would be suitable for numerical solution by a black-box ODE solver, e.g.~by any typical adaptive Runge-Kutta method \cite{AscherPetzold1998}, except that the constraint $0=g(\bq)$ is not, in fact, a differential equation at all.  Formulated as we have done it in cartesian coordinates, our problem is a \emph{differential-algebraic equation} (DAE) system.  This is interesting.


\section{Index and index reduction}  FIXME


\section{Stabilized index 2 formulation}  FIXME Exercise 9.10 in \cite{AscherPetzold1998}; see also \cite{Gearetal1985}


\section{Numerical solution}

FIXME Uses PETSc TS \cite{Balayetal2021,Bueler2021}, starting with BDF

\small

\bigskip
\bibliography{twoballs}
\bibliographystyle{siam}

\end{document}
