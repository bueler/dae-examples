\documentclass[letterpaper,final,12pt,reqno]{amsart}

\usepackage[total={6.0in,8.8in},top=1.3in,left=1.25in]{geometry}

\usepackage{times,bm,bbm,empheq,verbatim,fancyvrb,graphicx}
\usepackage[dvipsnames]{xcolor}

\usepackage[kw]{pseudo}

\pseudoset{left-margin=15mm,topsep=5mm,idfont=\texttt}

\usepackage{tikz}
\usetikzlibrary{decorations.pathreplacing}

% hyperref should be the last package we load
\usepackage[pdftex,
colorlinks=true,
plainpages=false, % only if colorlinks=true
linkcolor=blue,   % ...
citecolor=Red,    % ...
urlcolor=black    % ...
]{hyperref}

\DefineVerbatimEnvironment{cline}{Verbatim}{fontsize=\small,xleftmargin=5mm}

\renewcommand{\baselinestretch}{1.05}

\newtheorem{lemma}{Lemma}
\newtheorem*{example}{Example}

\newcommand{\Matlab}{\textsc{Matlab}\xspace}
\newcommand{\eps}{\epsilon}
\newcommand{\lam}{\lambda}
\newcommand{\RR}{\mathbb{R}}

\newcommand{\grad}{\nabla}
\newcommand{\Div}{\nabla\cdot}
\newcommand{\trace}{\operatorname{tr}}

\newcommand{\hbn}{\hat{\mathbf{n}}}

\newcommand{\bb}{\mathbf{b}}
\newcommand{\be}{\mathbf{e}}
\newcommand{\bbf}{\mathbf{f}}
\newcommand{\bg}{\mathbf{g}}
\newcommand{\bn}{\mathbf{n}}
\newcommand{\bq}{\mathbf{q}}
\newcommand{\br}{\mathbf{r}}
\newcommand{\bs}{\mathbf{s}}
\newcommand{\bt}{\mathbf{t}}
\newcommand{\bu}{\mathbf{u}}
\newcommand{\bv}{\mathbf{v}}
\newcommand{\bw}{\mathbf{w}}
\newcommand{\bx}{\mathbf{x}}
\newcommand{\by}{\mathbf{y}}
\newcommand{\bz}{\mathbf{z}}

\newcommand{\bF}{\mathbf{F}}
\newcommand{\bV}{\mathbf{V}}
\newcommand{\bX}{\mathbf{X}}

\newcommand{\bxi}{\bm{\xi}}

\newcommand{\blambda}{\bm{\lambda}}
\newcommand{\bzero}{\bm{0}}

\newcommand{\rhoi}{\rho_{\text{i}}}
\newcommand{\ip}[2]{\left<#1,#2\right>}

\newcommand{\Rpr}{R_{\text{pr}}}
\newcommand{\Rin}{R_{\text{in}}}
\newcommand{\Rfw}{R_{\text{fw}}}


\begin{document}
\title[Two balls, rigidly-connected]{Two balls, rigidly-connected: \\ a DAE case study}

\author{Ed Bueler}

\begin{abstract}
The problem of two equal-mass balls, rigidly connected by a massless rod, is described by an index-3 differential-algebraic equations (DAE) system.  An index-reduction procedure rewrites it as a stabilized index-2 DAE system.  Numerical solutions using implicit PETSc TS solvers are evaluated.
\end{abstract}

\maketitle

%\tableofcontents

\thispagestyle{empty}
\bigskip

\section{What are the cartesian equations of motion?}

Consider the problem of two equal masses $m$, labeled ``$a$'' and ``$b$'', moving in the $(x,y)$ plane, with $x$ horizontal and $y$ vertical.  We may form a column vector from their cartesian coordinates,
\begin{equation}
\bq(t) = \begin{bmatrix} q_1(t) \\ q_2(t) \\ q_3(t) \\ q_4(t) \end{bmatrix} = \begin{bmatrix} x_a(t) \\ y_a(t) \\ x_b(t) \\ y_b(t) \end{bmatrix}. \label{position}
\end{equation}

Now suppose the two masses move according to two kinds of forces.  First, gravity acts vertically downwards.  Second, suppose the masses are connected by a rigid, massless rod with length $\ell$.  The positions of the two masses are therefore constrained to satisfy $(x_a - x_b)^2 + (y_a - y_b)^2 = \ell^2$ at all times.  Equivalently, a certain scalar function is identically zero:
\begin{equation}
0 = g(\bq) = \frac{1}{2} \Big((q_1 - q_3)^2 + (q_2 - q_4)^2 - \ell^2\Big). \label{constraint}
\end{equation}
(The overall constant $\frac{1}{2}$ is chosen for later convenience.)

Physically speaking, the rod exerts a tension or expansion force along the line between the masses, which varies during the motion.  We denote this scalar force by $\lambda$, positive when the rod is pulling the two masses together.  Newton's second law says, of course, that the \emph{total} force $\bm{\Phi}(\bq,\lambda)$ determines the accelerations $\ddot \bq$, namely
\begin{equation}
m \ddot \bq = \bm{\Phi}(\bq,\lambda) \label{newtonssecond}
\end{equation}
describes the motion.  This is all fine and good, except how are we to determine the total (vector) forces including those enforcing the constraint?  In particular, how does $\bm{\Phi}$ depend on the tension $\lambda$, and how does the constraint \eqref{constraint} determine $\lambda$?


\section{As a rigid body: easy}

Classical mechanics does not regard this as a difficult problem.  It can be solved by a simple, planar application of the theory of \emph{rigid bodies} \cite[Vol.~1, Chapter 18]{Feynman2011i}.

Let $\xi(t),\eta(t)$ be the coordinates of the center of mass; this is the center of the rod.   Let $\theta(t)$ be the angle in radians between the rod and the horizontal axis, specifically the angle between the positive $x$-axis and the vector from the center of mass $(\xi,\eta)$ to the first mass $(x_a,y_a) = (q_1,q_2)$.  In terms of the three new variables $\xi,\eta,\theta$, we can write
\begin{equation}
\bq = \begin{bmatrix} \xi + (\ell/2) \cos\theta \\ \eta + (\ell/2) \sin\theta \\ \xi - (\ell/2) \cos\theta \\ \eta - (\ell/2) \sin\theta  \end{bmatrix}.  \label{cartesianrigid}
\end{equation}
Immediately constraint \eqref{constraint} is satisfied.

In the theory of rigid bodies the motion of the center of mass is determined by the external forces, here just gravity.  On the other hand the moment of inertial ($m\ell$) times the angular acceleration equals the torques applied to the rigid body.  Here there are no torques at all, so the angular acceleration is zero.  Thus, now that we are using the ``correct'' coordinates, the equations of motion are
\begin{subequations}
\label{rigidequations}
\begin{align}
      2 m \ddot\xi &= 0 \\
     2 m \ddot\eta &= -2 m g_r \\
m \ell \ddot\theta &= 0
\end{align}
\end{subequations}
As a first-order system this has dimension 6.

This is a delightfully trivial ODE system, with exact solution
\begin{subequations}
\label{rigidsolution}
\begin{align}
   \xi(t) &= \xi(0) + \dot\xi(0) t \\
  \eta(t) &= \eta(0) + \dot\eta(0) t - \frac{1}{2} g_r t^2 \\
\theta(t) &= \theta(0) + \dot\theta(0) t
\end{align}
\end{subequations}

The problem is thus easy in the correct coordinates.  However, we are interested in the solution of the DAE system in the original cartesian coordinates, primarily because other more-complicated problems do not have a cute theory like rigid bodies.  The rigid body exact solution \eqref{rigidsolution} will be used to verify the numerical solution of the ``wrong coordinates'' DAE solution.

In such verification usage we will need to convert the cartesian initial conditions to the initial conditions for the rigid body solution \eqref{rigidsolution}.  The key equations say that the cartesian velocity of each mass is the sum of the center-of-mass velocity plus the cross product of the angular velocity and the position vector of the mass relative to the center of mass, for instance $\bv = \bV + \bm{\omega}\times \br$ in obvious notation.  Thus, to do the initial-value conversion, we use \eqref{cartesianrigid} plus these equations for initial velocity:
\begin{subequations}
\label{velocitycartesianrigid}
\begin{align}
\dot q_1 &= \dot\xi - \dot\theta (q_2 - \eta) \\
\dot q_2 &= \dot\eta + \dot\theta (q_1 - \xi) \\
\dot q_3 &= \dot\xi - \dot\theta (q_4 - \eta) \\
\dot q_4 &= \dot\eta + \dot\theta (q_3 - \xi)
\end{align}
\end{subequations}


\section{Construction of a DAE system using Lagrangian dynamics}

Returning to our naive, cartesian problem, it is well-known that Newton's laws are poorly-suited to describing the forces in such a constrained situation, but that we can find the motion via \emph{Lagrangian dynamics}.  One can derive the Lagrangian approach from the calculus of variations, via Hamilton's principle of least action or the principle of virtual work \cite{Lanczos1970}, but here we will simply use the derived Euler-Lagrange equations (below) as the source of our DAE system.  This Lagrangian analytical approach applies broadly, across all of classical dynamics \cite{Layton1998}.

First, for unconstrained, non-dissipative motion described by the position variables $\bq(t) \in \RR^n$ and velocities $\dot\bq(t)$, one defines the \emph{Lagrangian} $\mathcal{L}_0(\bq,\dot\bq) = T(\bq,\dot\bq) - U(\bq)$ as the difference of kinetic and potential energy.  The motion $\bq(t)$ then solves the \emph{Euler-Lagrange differential equations} ($i=1,\dots,n$):
\begin{equation}
\frac{d}{dt} \frac{\partial \mathcal{L}}{\partial \dot q_i} = \frac{\partial \mathcal{L}}{\partial q_i}. \label{eulerlagrange}
\end{equation}
For constrained motion we must, however, modify the Lagrangian.  Suppose $\bg(\bq) \in \RR^k$ is a column vector of the constraints, i.e.
\begin{equation}
\bg(\bq)=\bzero. \label{generalconstraints}
\end{equation}
Now we define a vector of Lagrange multipliers $\blambda(t) \in \RR^k$ of the same length and a modified Lagrangian \cite[equation (58.2)]{Lanczos1970}
\begin{align}
\mathcal{L}(\bq,\dot\bq,\blambda) &= \mathcal{L}_0(\bq,\dot\bq) - \blambda^\top \bg(\bq)  \notag \\
  &= T(\bq,\dot\bq) - U(\bq) - \blambda^\top \bg(\bq). \label{extendedlagrangian}
\end{align}
Then the motion $\bq(t),\blambda(t)$ satisfies the system \eqref{eulerlagrange} for Lagrangian $\mathcal{L}$, along with the constraint equations \eqref{constraint}.  Physically, $\blambda$ is the (vector) force which enforces the constraints.

Because $\mathcal{L}$ does not depend on $\dot\blambda$, one may recover the constraints by the notional Euler-Lagrange equation $\bzero = d/dt(\partial \mathcal{L}/\partial \dot\blambda) = \partial \mathcal{L}/\partial\blambda = \bg(\bq)$.  In this sense the multipliers are treated as coordinates and the Euler-Lagrange equations include the constraint equations.

In our two-masses problem with $n=4$ there is $k=1$ (scalar) constraint $g(\bq)$ and thus one Lagrange multiplier $\lambda$, and it is the rod tension which we seek.  Based on the usual cartesian formula for kinetic energy and on gravitational potential energy, we have
\begin{equation}
T(\bv) = \frac{m}{2} \left(v_1^2+v_2^2+v_3^2+v_4^2\right), \qquad U(\bq) = m g_r \left(q_2+q_4\right), \label{energies}
\end{equation}
where $\bv(t) = \dot\bq(t)$ and $g_r>0$ is the acceleration of gravity.  The modified Lagrangian is
\begin{align}
\mathcal{L}(\bq,\bv,\lambda) &= T(\bv) - U(\bq) - \lambda g(\bq) \label{lagrangian} \\
  &= \frac{m}{2} \left(v_1^2+v_2^2+v_3^2+v_4^2\right) - m g_r \left(q_2+q_4\right) - \frac{\lambda}{2} \Big((q_1 - q_3)^2 + (q_2 - q_4)^2 - \ell^2\Big). \notag
\end{align}

From the Euler-Lagrange equations \eqref{eulerlagrange} and the constraint equation \eqref{constraint}, our two-masses problem is a system of 9 equations with (at most) first-order derivatives once we include the definition of the velocities:
\begin{subequations}
\label{rawsystem}
\begin{align}
  \dot q_1 &= v_1 \\
  \dot q_2 &= v_2 \\
  \dot q_3 &= v_3 \\
  \dot q_4 &= v_4 \\
m \dot v_1 &= - \lambda (q_1 - q_3) \\
m \dot v_2 &= - m g_r - \lambda (q_2 - q_4) \\
m \dot v_3 &= \lambda (q_1 - q_3) \\
m \dot v_4 &= - m g_r + \lambda (q_2 - q_4) \\
         0 &= \frac{1}{2} \Big((q_1 - q_3)^2 + (q_2 - q_4)^2 - \ell^2\Big) \label{rawsystem:constraint}
\end{align}
\end{subequations}

Next we rewrite system \eqref{rawsystem} using the vector notation from \cite[equation (9.30)]{AscherPetzold1998}:
\begin{subequations}
\label{system}
\begin{align}
\dot \bq &= \bv  \label{system:dotq} \\
m \dot \bv &= \bbf - \lambda\, G(\bq)^\top  \label{system:dotv} \\
0 &= g(\bq)  \label{system:constraint}
\end{align}
\end{subequations}
The column vector $\bbf = [0,-mg_r,0,-mg_r]^\top$ is the (constant) external force and the Jacobian matrix $G$ is a row vector
\begin{equation}
G(\bq) = \begin{bmatrix} {\displaystyle \frac{\partial g}{\partial q_i}} \end{bmatrix} = \begin{bmatrix} q_1-q_3, & q_2-q_4, & -(q_1-q_3), & -(q_2-q_4) \end{bmatrix}. \label{constraintjacobian}
\end{equation}
(In general $G$ is a $k\times n$ matrix.  Reference \cite{AscherPetzold1998} allows a nontrivial mass matrix $M(\bq)$, but here $M(\bq) = mI$.)  For future reference we calculate that $G(\bq) \bbf = 0$ and that
\begin{equation}
G(\bq) G(\bq)^\top = 2 (q_1-q_3)^2 + 2 (q_2 - q_4)^2 = 2 \ell^2 > 0.  \label{ggpd}
\end{equation}

System \eqref{system} would be suitable for numerical solution by a black-box ODE solver, e.g.~any adaptive explicit Runge-Kutta method \cite{AscherPetzold1998}, except that the constraint $0=g(\bq)$ is not, in fact, a differential equation at all.  Formulated as we have done it in cartesian coordinates, our problem is a \emph{differential-algebraic equation} (DAE) system.  Equations \eqref{system:dotq}, \eqref{system:dotv}, equivalently the first eight equations in system \eqref{rawsystem}, are differential equations, but \eqref{system:constraint} is algebraic.  Application of an implicit solver, one which consistently solves the discretized equations at each time step, will be necessary.


\section{DAE index}

Some DAE systems are more difficult to solve than others, but the various possibilities are neither simple to describe nor particularly easy to categorize.  One quantification which has proven value is the (\emph{differential}) \emph{index} of the DAE system.  This nonnegative integer is the minimum number of times that the algebraic equations, the constraints, must be differentiated in time before substitutions reveal an ODE system.  (An ODE system is thus an index-0 DAE system.)  While this is not obviously a rigorous definition, the linear case is, at least, precise \cite[Chapter IV.5]{HairerWanner1996}.

Our DAE problem \eqref{system} is a well-known type of mechanical system with equality constraints on the position variables (``holonomic constraints'' \cite{Lanczos1970}), a class of DAE problems with index 3.  Note that indices higher than 2 are traditionally called ``high index,'' because much less is known about reliable numerical solvers beyond index 2.

Let us show, by differentiating $g(\bq)=0$ with respect to time, why system \eqref{system} has index 3.  (The argument here will not exclude generating an ODE in fewer differentiations; this shows that the differential index is at most 3.)  Differentiating \eqref{system:constraint} once with respect to $t$ gives
\begin{align}
0 &= \frac{d}{dt} \bigg(\frac{1}{2} \Big((q_1 - q_3)^2 + (q_2 - q_4)^2 - \ell^2\Big)\bigg) \notag \\
  &= (q_1 - q_3)(v_1 - v_3) + (q_2 - q_4) (v_2 - v_4). \label{rawvelocityconstraint}
\end{align}
By substituting \eqref{system:dotq} and \eqref{constraintjacobian}, we may write this as a matrix-vector product:
\begin{equation}
0 = G(\bq) \bv. \label{velocityconstraint}
\end{equation}
This equation is called the \emph{velocity constraint}.

Differentiating again, and then substitution of \eqref{system:dotv} and \eqref{system:constraint}, yields the \emph{acceleration constraint} \cite{Layton1998}
\begin{equation}
0 = \frac{d}{dt} \Big((q_1 - q_3)(v_1 - v_3) + (q_2 - q_4) (v_2 - v_4)\Big) = (v_1 - v_3)^2 + (v_2 - v_4)^2 - \lambda \frac{2 \ell^2}{m} \label{rawddconstraint}
\end{equation}
or equivalently
\begin{equation}
\lambda = \frac{m}{2\ell^2} \left((v_1 - v_3)^2 + (v_2 - v_4)^2\right). \label{rawlambda}
\end{equation}
Writing this using matrix-vector notation is not so obvious.  However, since $G(\bq)\bv = \sum_i G_{1i}(\bq) v_i$, substituting \eqref{system:dotv} yields
\begin{align}
0 &= \sum_{i,j=1}^4 \frac{\partial G_{1i}(\bq)}{\partial q_j} \dot q_j v_i + \sum_{i=1}^4 G_{1i}(\bq) \dot v_i = \bv^\top \frac{\partial G(\bq)}{\partial \bq} \bv + G(\bq) \frac{1}{m} \left(\bbf - \lambda\, G(\bq)^\top\right) \notag \\
  &= \frac{\partial (G(\bq)\bv)}{\partial \bq} \bv +  \frac{G(\bq)}{m} \bbf - \lambda \frac{G(\bq) G(\bq)^\top}{m} = \frac{\partial (G(\bq)\bv)}{\partial \bq} \bv - \lambda \frac{G(\bq) G(\bq)^\top}{m}.  \label{ddconstraint}
\end{align}
Here
\begin{equation}
\left(\frac{\partial G(\bq)}{\partial \bq}\right)_{ij} = \frac{\partial G_{1i}(\bq)}{\partial q_j} \qquad \text{and} \qquad
\left(\frac{\partial (G(\bq)\bv)}{\partial \bq}\right)_{j} = \frac{\partial G_{1i}(\bq)}{\partial q_j} v_j,
\end{equation}
with the latter regarded as a row vector.  In fact, let
\begin{equation}
H(\bv) = \frac{\partial (G(\bq)\bv)}{\partial \bq} = \begin{bmatrix} v_1-v_3, & v_2-v_4, & -(v_1-v_3), & -(v_2-v_4) \end{bmatrix}. \label{velocityconstraintderiv}
\end{equation}
Applying \eqref{ggpd} in \eqref{ddconstraint} and solving for $\lambda$ gives \eqref{rawlambda} again:
\begin{equation}
\lambda = \frac{m}{2\ell^2} H(\bv) \bv = \frac{m}{2\ell^2} \left((v_1 - v_3)^2 + (v_2 - v_4)^2\right). \label{lambda}
\end{equation}

Differentiating \eqref{lambda}, and using this formula for $\dot\lambda$ to replace \eqref{system:constraint}, finally converts system \eqref{system} into an ODE system.  This \emph{unstabilized index reduction} shows that \eqref{system} has at most index 3.  However, we do not actually need or want the final ODE system, or even the differential equation for $\dot \lambda$.  Instead, we will use a stabilized index 2 DAE formulation as described in the next section.


\section{Stabilized index-2 formulation}

\begin{quote}
\emph{For a DAE of index greater than 2 it is usually best to use one of the index-reduction techniques \dots to rewrite the problem in lower-index form.} \, \cite[p 262]{AscherPetzold1998}
\end{quote}

The approach of Gear and others \cite{Gearetal1985} is to replace the original index-3 DAE system \eqref{system} with one which is more constrained, and has lower index, but which remains a DAE.  This involves two changes to system \eqref{system}.  First we append the velocity constraint \eqref{velocityconstraint} to \eqref{system}.  Then, to compensate, we add a corresponding Lagrange multiplier $\mu$, and use it to add a restoring constraint force to equation \eqref{system:dotq}:
\begin{subequations}
\label{stab}
\begin{align}
\dot \bq &= \bv - \mu\, G(\bq)^\top \label{stab:dotq} \\
m \dot \bv &= \bbf - \lambda\, G(\bq)^\top  \label{stab:dotv} \\
0 &= g(\bq)  \label{stab:qconstraint} \\
0 &= G(\bq) \bv  \label{stab:vconstraint}
\end{align}
\end{subequations}

As shown momentarily, system \eqref{stab} has index 2.  It is called the \emph{stabilized index-2 formulation} \cite[Exercise 9.10]{AscherPetzold1998} of our index-3 constrained mechanical system.  When such mechanical systems initially have $n$ position variables and $k$ constraints, thus total dimension $2n+k$, the dimension of the stabilized index-2 formulation becomes $2(n+k)$.  In our case $n=4$ and $k=1$.

It is easy to see that the exact solution is unchanged.  In fact, the solution of \eqref{stab} is a quadruple $\bq(t),\bv(t),\lambda(t),\mu(t)$ in which the triple $\bq(t),\bv(t),\lambda(t)$ solves \eqref{system} and $\mu(t)=0$ identically.  To show this, differentiate \eqref{stab:qconstraint} with respect to time and apply \eqref{ggpd}, \eqref{stab:dotq}, \eqref{stab:vconstraint}:
\begin{equation}
0 = G(\bq) \dot \bq = G(\bq) \left(\bv - \mu G(\bq)^\top\right) = - 2 \ell^2 \mu.
\end{equation}
When system \eqref{stab} is solved exactly it follows that $\mu(t)=0$.  However, the modified velocity equation \eqref{stab:dotq} will assist the numerical solver in staying on the constraint, making the numerical solution to \eqref{stab} superior the unstabilized formulation \eqref{system}.

To further expose the structure of system \eqref{stab}, let
\begin{equation}
\bx = \begin{bmatrix} \bq \\ \bv \end{bmatrix} \qquad \text{and} \qquad \bz = \begin{bmatrix} \mu \\ \lambda \end{bmatrix}.
\end{equation}
In this notation, system \eqref{stab} has the form
\begin{subequations}
\label{hessen}
\begin{align}
\dot \bx &= \br(\bx,\bz) \label{hessen:differential} \\
  \bzero &= \bs(\bx) \label{hessen:algebraic}
\end{align}
\end{subequations}
The $\bx$ variables are \emph{differential}, as they have time derivatives, while the $\bz$ variables are \emph{algebraic}.  For such a DAE system one would convert to an ODE system by computing by differentiating \eqref{hessen:algebraic} with respect to $t$ and then substituting \eqref{hessen:differential},
\begin{equation}
\bzero = \frac{\partial \bs}{\partial \bx} \dot \bx = \frac{\partial \bs}{\partial \bx} \br.
\end{equation}
This equation which is algebraic; note $\partial \bs/\partial \bx$ is a $2k\times 2n = 2\times 8$ matrix.  A second time-differentiation is now needed, yielding
\begin{equation}
\bzero = \frac{\partial^2 \bs}{\partial \bx^2}\left[\br, \br\right] + \frac{\partial \bs}{\partial \bx} \frac{\partial \br}{\partial \bx}\, \br + \frac{\partial \bs}{\partial \bx} \frac{\partial \br}{\partial \bz}\, \dot \bz  \label{hessen:indextwo}
\end{equation}
wherein the second derivative $\partial^2 \bs/\partial \bx^2$ is tensor-valued.  This gets us to the main point: equation \eqref{hessen:indextwo} can be solved for the needed time-derivative $\dot \bz$ if the matrix sitting beside it is invertible.  That is, system \eqref{hessen} has index 2 as long as
\begin{equation}
B = \frac{\partial \bs}{\partial \bx} \frac{\partial \br}{\partial \bz} \in \RR^{2k\times 2k} \quad \text{ is invertible}. \label{hessen:criterion}
\end{equation}
In general, when \eqref{hessen:criterion} holds we say system \eqref{hessen} is a \emph{Hessenberg} (or \emph{pure}) index-2 system.

In our two-balls problem one differentiates system \eqref{stab} to find $B$, a $2\times 2$ matrix which is in fact diagonal:
\begin{equation}
B = \begin{bmatrix}  G(\bq) G(\bq)^\top & 0 \\ 2 G(\bq) \bv & m^{-1} G(\bq) G(\bq)^\top \end{bmatrix} = 2\ell^2 \begin{bmatrix}  1 & 0 \\ 0 & m^{-1} \end{bmatrix}.
\end{equation}
As this matrix is invertible, we have shown that system \eqref{stab} has index 2.  Thus our index-reduction and stabilization procedure has converted the original index-3 problem into a Hessenberg index-2 system.


\section{Numerical solutions}

\begin{quote}
\emph{[For] Hessenberg index-2 DAEs, the coefficients of the multistep methods must satisfy a set of order conditions which is in addition to the order conditions for ODEs, to attain order greater than 2.  It turns out that these additional order conditions are satisfied by BDF methods.} \, \cite[p 267]{AscherPetzold1998}
\end{quote}

Our numerical implementation in this section will solve initial-value problems for a first-order, index-2 DAE system with 10 scalar variables.  We will apply the TS (time-steppers) component of the PETSc library \cite{Balayetal2021,Bueler2021} to solve the DAE system.

To use the fully-implicit solvers in PETSc TS, specifically BDF, the problem is best put in the most general DAE form
\begin{equation}
\bF(t,\bu,\bu')=0. \label{fullyimplicit}
\end{equation}
We choose to order the components of $\bu$ as follows:
\begin{equation}
\bu = \begin{bmatrix} \bq \\ \bv \\ \mu \\ \lambda \end{bmatrix}
= \begin{bmatrix} q_1 & q_2 & q_3 & q_4 & v_1 & v_2 & v_3 & v_4 & \mu & \lambda \end{bmatrix}^\top
\end{equation}
The function $\bF$ is defined following system \eqref{stab}:
\begin{equation}
\bF(t,\bu,\dot\bu)
 = \begin{bmatrix}
\dot \bq - \bv + \mu\, G(\bq)^\top \\
m \dot \bv - \bbf + \lambda\, G(\bq)^\top \\
g(\bq) \\
G(\bq) \bv
 \end{bmatrix}
 = \begin{bmatrix}
  \dot q_1 - v_1 + \mu (q_1 - q_3) \\
  \dot q_2 - v_2 + \mu (q_2 - q_4) \\
  \dot q_3 - v_3 - \mu (q_1 - q_3) \\
  \dot q_4 - v_4 - \mu (q_2 - q_4) \\
m \dot v_1 + \lambda (q_1 - q_3) \\
m \dot v_2 + m g_r + \lambda (q_2 - q_4) \\
m \dot v_3 - \lambda (q_1 - q_3) \\
m \dot v_4 + m g_r - \lambda (q_2 - q_4) \\
\frac{1}{2} \Big((q_1 - q_3)^2 + (q_2 - q_4)^2 - \ell^2\Big) \\
(q_1 - q_3) (v_1 - v_3) + (q_2 - q_4) (v_2 - v_4)
\end{bmatrix}
\end{equation}
Note that $\bF$ does not explicitly depend on $t$.

At each step of an implicit method, a system of 10 nonlinear equations will be solved.  While this can be done using a finite-difference Jacobian (see the next section), it is most efficient using an exact, analytical Jacobian.  Effectively this means we must calculate two $10 \times 10$ matrices, next, the derivatives of $\bF$ with respect to $\bu$ and $\dot\bu$.

The matrix $\partial\bF/\partial\bu$ has nested block structure with a larger upper-left $8\times 8$ block associated to the $\bq,\bv$ variables and a smaller lower-right $2\times 2$ block which is identically zero:
\begin{equation}
\frac{\partial\bF}{\partial\bu} =
\left[\begin{array}{cc|cc}
 \mu C & -I     & G(\bq)^\top & \\
\lam C &        &             & G(\bq)^\top \\ \hline
G(\bq) &        &             & \\
H(\bv) & G(\bq) &             &
\end{array}\right].
\end{equation}
(Blank blocks/entries are zero.)  Here $I$ is the $4\times 4$ identity matrix and
\begin{equation}
C = \begin{bmatrix}
 1 &    & -1 & \\
   &  1 &    & -1 \\
-1 &    &  1 & \\
   & -1 &    &  1
\end{bmatrix}.
\end{equation}
Recall that $H(\bv)$ is the $1\times 4$ row matrix defined in \eqref{velocityconstraintderiv}.

Because our problem is a nontrivial DAE, the matrix $\partial\bF/\partial\dot\bu$ is singular (nullity $=2$), but also conveniently diagonal:
\begin{equation}
\frac{\partial\bF}{\partial\dot\bu} = \left[\begin{array}{cc|cc}
I &    &   & \\
  & mI &   & \\ \hline
  &    & 0 & \\
  &    &   & 0
\end{array}\right]
\end{equation}

The PETSc implicit TS solvers use a callback for the function $\bF(t,\bu,\dot\bu)$  and, because implicit discrete methods generate a linear system based on a combined Jacobian \cite[section 2.5]{Balayetal2021}
\begin{equation}
J_\sigma = \sigma \frac{\partial\bF}{\partial\dot\bu} + \frac{\partial\bF}{\partial\bu} = \left[\begin{array}{cc|cc}
\sigma I + \mu C & -I     & G(\bq)^\top & \\
\lam C & \sigma m I &             & G(\bq)^\top \\ \hline
G(\bq) &        & 0           & \\
H(\bv) & G(\bq) &             & 0
\end{array}\right].
\end{equation}
for some scalar $\sigma$.  A key idea is that the DAE itself, and many implicit methods, will be uniquely solvable because $J_\sigma$ is nonsingular for $\sigma > 0$.

Next we need specific examples for testing purposes.

\subsection*{Example A}  Suppose the masses are tennis balls ($m=58\,\text{g}$), the rod has length $\ell=50\,\text{cm}$, and we are on Earth ($g_r=9.81\,\text{m}\,\text{s}^{-2}$).  For Example A we take the initial positions to be $(q_1,q_2)=(0,1)\,\text{m}$ and $(q_3,q_4)=(0,1.5)\,\text{m}$, thus the two balls start one above the other, with the lower one meter off the ground.  The initial velocities are $(v_1,v_2)=(10,10)\,\text{m}\,\text{s}^{-1}$ and $(v_3,v_4)=(15,10)\,\text{m}\,\text{s}^{-1}$ so that the second, higher mass starts at a higher velocity.  Observe that both constraints, $0=g(\bq)$ and $0=G(\bq)\bv$, are satisfied for these initial conditions.  That is, the initial positions are distance $\ell$ apart and the velocity constraint \eqref{stab:vconstraint} is satisfied.

\subsection*{Example B}  FIXME suppose balls falling straight down, side-by-side $\ell$ apart; $\partial\bF/\partial\bu$ clearly not invertible because $\mu=\lambda=0$; of course $\partial\bF/\partial\dot\bu$ also not invertible; but $J_\sigma$ invertible

FIXME start with BDF

\small

\bigskip
\bibliography{twoballs}
\bibliographystyle{siam}

\end{document}
